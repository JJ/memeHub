\chapter{Implementación}

La implementación del software se ha dividido en hitos. Estos, han sido definidos en Github
y cada uno de ellos contiene un grupo de \textit{issues} que se corresponden con las distintas
mejoras que se han ido incorporando al software a lo largo de su desarrollo.\\

Esquema para luego redactar bien:

- Empezamos aprendiendo React

- Hice una demo básica

- Busqué librerias y cosas mientras seguí aprendiendo React

- Despues de unos cuantos proyectos de prueba y elegidas las librerias me lancé con el editor

- A partir de la demo rehice todo para adaptarlo a la libería 

- Lo primero fue hacer que se pudiera añadir una foto base (DragAndDrop y Konva canvas)

- Explicar todo tema lienzo, cálculos etc..

- Se añadió poder quitar la foto y volver a añadir otra

- Comenzamos el poder añadir texto

- todo tema texto, textarea, propiedades, drag, transformador, lo de adaptar el texto al tamaño de la caja bla bla 

- Comienzo de la toolbar para poder poner el texto bonito

- Tema toolbar, material-ui, como mandamos la info desde la toolbar al texto y lo actualizamos bla bla

- Hablar de ajustes varios, tema eventos etc.. clicks handlers bla bla

- Añadido el ctrl + v de imagen principal (esto quizá explicarlo después para juntar todo el tema texto)

- Handleo de varios textos diferentes

- Cuando ya se cambiaban los textos desde la toolbar ahora como se cambia la toolbar para que coincida con las opciones del texto

- Hablar un poco de tema componentes funcionales vs clases (todos los cambios que se hicieron)

- PRIMERA TANDA DE TESTS (preguntar a JJ si explicar esto todo luego o aquí)

- Además de texto ahora a añadir mas cosas, Imágenes y sus transformaciones

- Hablar del tema lineas, como se movian las cosas al pintar encima, la lógica de las herramientas
  como se ha solucionado, porque se ha hecho así y el React.clone()

- [pensar lo que falta por aqui]

- tema eventos, copiar pegar, guardar etc